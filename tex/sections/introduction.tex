\chapter{Introduction}
\label{ch:Introduction}

Books are, even in times of blogs and electronic publications\footnote{This is 2016. It is likely that this will change.}, the most important way of to publish fictional and non-fictional stories, reports, education material, well-collected scientific material and lifetime achievements. The writing within these books is material created at a certain point of history within a specific society under concrete, but not always known, circumstances. This is reflected in the usage of grammar and vocabulary. This reflection can be seen when reading a specific book, but it could also seen in a statistical way when analysing many books. A major barrier is that someone needs to read these books and at the same time is actively looking for specific patterns, until recently. Google as one of the big companies that handle data was able to collect the content of a variety of books within their Google Books\footnote{\url{https://books.google.com/}} project. They than analysed the content and collected statistical information about chains of words, called n-Grams (\cite{Google_nGrams}). The information is presented as time series which store the usage count of specific n-Grams over years. This enables people to check assumptions about how words were used over time and they are able to use these information to conclude about history of concrete phrases and abstract concepts. The data already helped other groups to extract valuable knowledge in linguistics (\cite{others1,others5,others7}), NLP for video descriptions (\cite{others2}), probability research (\cite{others3,others6}), investigation of cultural change (\cite{others4}) and to improve spell and grammar checking (\cite{languagetool}).

A major question that arises when analysing single words or n-Grams is if they are influenced by others or if there is any connection between particular groups of words and phrases. There are two issues when someone's tries to answer this question. The first one is to find a scientific measure of \enquote{influence}. Without such concrete way people will be affected by confirmation bias which makes most of them rate what they see depending on what they expect. The second one is two come up with ideas which candidates could have a connection. People, no matter of which profession, have a framed, subjective view on the world and therefore are unlikely to start their research with unexpected result. The existence of both issues hinders researchers to apply the scientific method to some fields and let them rely on expert knowledge and skills.

In this work, we solve the first problem and the second one partly. We provide and discuss a metric of similarity which also enables users to only take parts of the whole time series into account. This may not be the only we to describe similarity and we want to point out that it is important to check whether our proposal is suited for your work before applying it. Naturally other fields of application may require small changes or totally different approaches on how to describe connections between time series. The similarity measure directly leads to a way of finding similar n-Grams to a given query. The reason why this solves the framing problem only partly is that a user still needs to start with a query and because a na{\"\i}ve search for neighbours is too slow when using the complete data set that Google provides.

To solve the second issue completely we need to speed-up the lookup-process. This issue here is that, compared to other time series data sets, the Google n-gram data is special which makes it unsuited for other indexing approaches. Therefore we explore our own way on how to store the time series data in a way that exploits similarities between them. This is done by transforming the data and then look for parts which can be shared between the transformed instances. It leads to an compression effect and speeds up query processing in a way that enable others then simple single-n-Gram queries. Also this enable researcher with low computational power and storage\footnote{compared to companies like Google} to process the whole data set of $1$ to $5$-grams.

Enabling fast search queries is not only an advantage for human users of the system. It also enables machines to gain information about the history of words and phrases and may enable novel techniques to visualize and analyse the data. The relationship of n-grams can then be expressed as pair-wise distance, as distance to carefully selected words, as graph with or without edge label or as multigraph which takes time slices into account. The fast on-demand availability of the similarity data can be key to a whole new kind of research work and, without doubt, will lead to interesting results.

An important disclaimer: the n-gram corpus used for this work makes the assumption that every usage of an n-gram has the same weight. This was already criticised by other researchers (\cite{countbad}) and we are aware of this problem. On the other hand, this data set is the only publicly accessible one of this kind and size and since there are not alternatives, we do not have a real choice. In principle our techniques also work with weighted sums which may be available in the future. We hope that large scale literature research becomes easier and we urge stakeholders to establish a workable fair use policy and to build up a digital archive which includes works from a wide variety of publishers and provides a proper API\@. Right now we are limited to bypass copyright limitations and rely on the, surely not selfless, kindness of big companies.

This work will first set a baseline by explaining data mangling and clean-up, the choice of a similarity measure and the reason why indexing this kind of data does not work with known methods. We then will present our method from its foundation to a concrete algorithm and optimizations as well as reasons why our original idea works worse than expected. Afterwards we will discuss implementation details and will run a full analysis and comparison with the baseline. At many points we present alternatives to our approach which we encourage the reader to take serious. We conclude this work with possible future research topics and an outlook on the benefits of data-driven research based on this type of data.
