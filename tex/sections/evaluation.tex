\chapter{Evaluation}
\label{ch:evaluation}

We now want to evaluate the different parts that we worked out before. We start with a definition:

\begin{definition}[Parameter\slash{}Filter response]
    A parameter or more specific a filter response describes how an adjustable variable influences a measurable value, e.g.:

    \begin{itemize}
        \item A filter influences the error made by an algorithm in a linear way, iff the error can be described as $f(x) = ax + b$ in terms of the filter value $x$ and two constants $a$ and $b$.\footnote{This is not the case for our algorithm.}
        \item The maximum error of the compressor has a monotonic response to the compression rate. This means that if the maximum error is increased, the compression rate does not decrease.\footnote{That is also not the case for our algorithm because of its greedy nature. It is true in when the response curve is smoothed though.} This does not apply any additional attributes like that the compression rate is continuous in terms of the maximum error nor that it can be even expressed as a function of it.
    \end{itemize}

    The response curve is the graphical representation of the measured value plotted as function of the adjustable parameter.
\end{definition}

Next we want to check if our baseline selection makes sense by check the neighborhood search for semantic meaning. Afterwards we check how our compression algorithm affects the DTW calculation. We continue with testing different parameters of our tracing algorithm and finish with some obligatory Performance measurements.



\section{Baseline Sanity}
\label{sec:evaluation:baseline}

\begin{table}[t]
    \begin{subtable}[t]{.45\textwidth}
        \centering
        \begin{tabularx}{\textwidth}{r R l}
            \toprule
            rank & $n$-gram & distance\\
            \midrule
            \num{0} & drug & \num{0}\\
            \num{1} & treatment & \num{0.0334646}\\
            \num{2} & impair & \num{0.0341294}\\
            \num{3} & fundament & \num{0.0346226}\\
            \num{4} & scar & \num{0.0348444}\\
            \num{5} & tract & \num{0.0351414}\\
            \num{6} & conjunct & \num{0.0352033}\\
            \num{7} & univers & \num{0.0352465}\\
            \num{8} & demonstr & \num{0.035316}\\
            \num{9} & western & \num{0.0353187}\\
            \num{10} & li & \num{0.0353931}\\
            \num{11} & mayb & \num{0.0354463}\\
            \num{12} & sex & \num{0.0354603}\\
            \num{13} & promot & \num{0.0354636}\\
            \midrule
            \num{14} & member & \num{0.0354739}\\
            \num{15} & blood & \num{0.0355806}\\
            \num{16} & action & \num{0.0357162}\\
            \num{17} & malign & \num{0.0357506}\\
            \num{18} & counter & \num{0.0357843}\\
            \num{19} & formal & \num{0.0358954}\\
            \bottomrule
        \end{tabularx}
        \caption{$[1,256]$}\label{tab:ranking_drug_all}
    \end{subtable}
    \hfill
    \begin{subtable}[t]{.45\textwidth}
        \centering
        \begin{tabularx}{\textwidth}{r R l}
            \toprule
            rank & $n$-gram & distance\\
            \midrule
            \num{0} & drug & \num{0}\\
            \num{1} & character & \num{0.0186669}\\
            \num{2} & abnorm & \num{0.0197631}\\
            \num{3} & recurr & \num{0.0197718}\\
            \num{4} & dissemin & \num{0.020063}\\
            \num{5} & retent & \num{0.0202669}\\
            \num{6} & primari & \num{0.0203592}\\
            \num{7} & transplant & \num{0.0204965}\\
            \num{8} & pregnant & \num{0.0206043}\\
            \num{9} & tension & \num{0.0207585}\\
            \num{10} & southeast & \num{0.020773}\\
            \midrule
            \num{11} & matern & \num{0.0208396}\\
            \num{12} & crucial & \num{0.0210149}\\
            \num{13} & region & \num{0.021018}\\
            \num{14} & clinic & \num{0.0210344}\\
            \num{15} & specif & \num{0.0210545}\\
            \num{16} & proxim & \num{0.0210849}\\
            \num{17} & relax & \num{0.0211782}\\
            \num{18} & morpholog & \num{0.0211811}\\
            \num{19} & vulner & \num{0.0212482}\\
            \bottomrule
        \end{tabularx}
        \caption{$[129,256]$}\label{tab:ranking_drug_secondhalf}
    \end{subtable}
    \caption[Neighbors of \enquote{drug}, full and half range]{Neighbors: $1$-grams, $r = 10$, \enquote{drug}}\label{tab:ranking_drug_sane}
\end{table}

Before we start to evaluate the actual algorithms, we show that our baseline is sane. \autoref{tab:ranking_drug_all} shows the top \num{20} neighbors of the $1$-gram \enquote{drug} with a warping radius of $r = 10$, measured over the full \num{256} years. The auto-cutoff-point as described in \autoref{ssec:baseline:sim:rank} is marked. For these examples, only neighbors from the $1$-gram data set are shown. Neighbors are represented by their normalized placeholder, which is derived as illustrated in \autoref{sec:baseline:data}. Therefore, they are not necessarily real words and can represent multiple original entries within the data set. For example, \enquote{treatment} represents \enquote{treatment}, \enquote{treatmental}, \enquote{treatmente}, and \enquote{treatments}. The neighbors are words that are related to the concept \enquote{drug} with some exceptions, for example the neighbor \enquote{li}, to which no other word was transformed during the normalization and stemming procedure. The Google data set just contains \enquote{li} as a word and we are not aware of a proper interpretation of this word. The meaningfulness increases when limiting the DTW to the second half of the data set. This is shown in \autoref{tab:ranking_drug_secondhalf}. It also turns out that \enquote{drug} is used in two contexts: as medicine and in terms of narcotics.

\begin{table}[t]
    \begin{subtable}[t]{.45\textwidth}
        \centering
        \begin{tabularx}{\textwidth}{r R l}
            \toprule
            rank & $n$-gram & distance\\
            \midrule
            \num{0} & drug & \num{0}\\
            \num{1} & treatment & \num{0.0516925}\\
            \num{2} & follow & \num{0.052971}\\
            \num{3} & new & \num{0.0534919}\\
            \num{4} & s & \num{0.0537821}\\
            \num{5} & enlarg & \num{0.0540329}\\
            \num{6} & press & \num{0.0540962}\\
            \num{7} & about & \num{0.0541345}\\
            \num{8} & use & \num{0.0541729}\\
            \midrule
            \num{9} & sever & \num{0.0541923}\\
            \num{10} & flow & \num{0.054294}\\
            \num{11} & low & \num{0.0544294}\\
            \num{12} & high & \num{0.0544914}\\
            \num{13} & avoid & \num{0.0545074}\\
            \num{14} & daili & \num{0.0545184}\\
            \num{15} & for & \num{0.054555}\\
            \num{16} & skill & \num{0.0546174}\\
            \num{17} & remov & \num{0.054691}\\
            \num{18} & chang & \num{0.0547283}\\
            \num{19} & time & \num{0.0547318}\\
            \bottomrule
        \end{tabularx}
        \caption{$r = 0$}\label{tab:ranking_drug_r0}
    \end{subtable}
    \hfill
    \begin{subtable}[t]{.45\textwidth}
        \centering
        \begin{tabularx}{\textwidth}{r R l}
            \toprule
            rank & $n$-gram & distance\\
            \midrule
            \num{0} & drug & \num{0}\\
            \num{1} & treatment & \num{0.0334636}\\
            \num{2} & fundament & \num{0.0335999}\\
            \num{3} & impair & \num{0.0339341}\\
            \midrule
            \num{4} & member & \num{0.0339427}\\
            \num{5} & demonstr & \num{0.0341886}\\
            \num{6} & scar & \num{0.0346951}\\
            \num{7} & region & \num{0.0347242}\\
            \num{8} & tract & \num{0.0348535}\\
            \num{9} & mayb & \num{0.0350965}\\
            \num{10} & becaus & \num{0.0351256}\\
            \num{11} & current & \num{0.0351304}\\
            \num{12} & degener & \num{0.0351406}\\
            \num{13} & malign & \num{0.035153}\\
            \num{14} & conjunct & \num{0.0351743}\\
            \num{15} & western & \num{0.0352145}\\
            \num{16} & univers & \num{0.0352465}\\
            \num{17} & blood & \num{0.035328}\\
            \num{18} & educ & \num{0.0353597}\\
            \num{19} & li & \num{0.0353931}\\
            \bottomrule
        \end{tabularx}
        \caption{$r = 20$}\label{tab:ranking_drug_r20}
    \end{subtable}
    \caption[Neighbors of \enquote{drug}, different values of $r$]{Neighbors: $1$-grams, $[1,256]$, \enquote{drug}}\label{tab:ranking_drug_rwrong}
\end{table}

The warping radius plays an important role when extracting good results from the data. \autoref{tab:ranking_drug_r0} and \autoref{tab:ranking_drug_r20} show too low and too high radius values and how in the first case the neighbors tend to be very pointless while in the latter case the results seem to be very common words.

\begin{table}[t]
    \begin{subtable}[t]{.45\textwidth}
        \centering
        \begin{tabularx}{\textwidth}{r R l}
            \toprule
            rank & $n$-gram & distance\\
            \midrule
            \num{0} & know & \num{0}\\
            \num{1} & do & \num{0.00973317}\\
            \num{2} & never & \num{0.010465}\\
            \num{3} & not & \num{0.0106112}\\
            \num{4} & what & \num{0.0111916}\\
            \num{5} & out & \num{0.0112691}\\
            \num{6} & how & \num{0.0114273}\\
            \num{7} & down & \num{0.0114877}\\
            \midrule
            \num{8} & no & \num{0.0116395}\\
            \num{9} & think & \num{0.0116744}\\
            \num{10} & own & \num{0.0123218}\\
            \num{11} & away & \num{0.0123529}\\
            \num{12} & mistak & \num{0.012384}\\
            \num{13} & that & \num{0.0125661}\\
            \num{14} & hear & \num{0.012628}\\
            \num{15} & as & \num{0.0126511}\\
            \num{16} & neither & \num{0.0126813}\\
            \num{17} & so & \num{0.0126882}\\
            \num{18} & one & \num{0.012772}\\
            \num{19} & too & \num{0.0127955}\\
            \bottomrule
        \end{tabularx}
        \caption{\enquote{know}}\label{tab:ranking_know_all}
    \end{subtable}
    \hfill
    \begin{subtable}[t]{.45\textwidth}
        \centering
        \begin{tabularx}{\textwidth}{r R l}
            \toprule
            rank & $n$-gram & distance\\
            \midrule
            \num{0} & war & \num{0}\\
            \num{1} & fight & \num{0.0215918}\\
            \num{2} & peac & \num{0.0238117}\\
            \num{3} & freedom & \num{0.024415}\\
            \num{4} & enemi & \num{0.0247643}\\
            \num{5} & impend & \num{0.0249975}\\
            \num{6} & steadi & \num{0.0250595}\\
            \midrule
            \num{7} & militari & \num{0.0253438}\\
            \num{8} & nation & \num{0.025674}\\
            \num{9} & peopl & \num{0.0262382}\\
            \num{10} & countri & \num{0.0262423}\\
            \num{11} & attack & \num{0.0263197}\\
            \num{12} & arm & \num{0.0264513}\\
            \num{13} & threaten & \num{0.0264571}\\
            \num{14} & ralli & \num{0.0264952}\\
            \num{15} & battl & \num{0.0265006}\\
            \num{16} & suppli & \num{0.0268525}\\
            \num{17} & violent & \num{0.0268842}\\
            \num{18} & forese & \num{0.0269896}\\
            \num{19} & over & \num{0.0270059}\\
            \bottomrule
        \end{tabularx}
        \caption{\enquote{war}}\label{tab:ranking_war_all}
    \end{subtable}
    \caption[Neighbors of \enquote{know} and \enquote{war}]{Neighbors: $1$-grams, $r = 10$, $[1,256]$}\label{tab:ranking_other1grams_sane}
\end{table}

Another interesting but logical observation is that common verbs like \enquote{know} result in unenlightening results as shown in \autoref{tab:ranking_know_all}. On the other hand nouns like \enquote{war} show sane and meaningful data. This is shown in \autoref{tab:ranking_war_all}.

\begin{table}[t]
    \begin{subtable}[t]{.45\textwidth}
        \centering
        \begin{tabularx}{\textwidth}{r R l}
            \toprule
            rank & $n$-gram & distance\\
            \midrule
            \num{0} & think therefore i am & \num{0}\\
            \num{1} & diversif from & \num{0.0674405}\\
            \num{2} & term leadership & \num{0.0749373}\\
            \num{3} & temporarili avoid & \num{0.0750777}\\
            \num{4} & fellow survivor & \num{0.0756024}\\
            \midrule
            \num{5} & relev result & \num{0.0756395}\\
            \num{6} & from subpoena & \num{0.0764528}\\
            \num{7} & yet unsur & \num{0.0765537}\\
            \num{8} & generat clear & \num{0.0767518}\\
            \num{9} & arson charg & \num{0.0767596}\\
            \num{10} & fieldwork have & \num{0.0768322}\\
            \num{11} & cultur despit & \num{0.0770261}\\
            \num{12} & fewer manag & \num{0.0770723}\\
            \num{13} & poor weight & \num{0.0771743}\\
            \num{14} & fund student & \num{0.0772}\\
            \num{15} & stanley miller & \num{0.0774633}\\
            \num{16} & requir technolog & \num{0.0775335}\\
            \num{17} & particip keep & \num{0.0775552}\\
            \num{18} & into gene & \num{0.0775731}\\
            \num{19} & saxophon to & \num{0.0775923}\\
            \bottomrule
        \end{tabularx}
        \caption{\enquote{think therefore i am}}\label{tab:ranking_chance_1}
    \end{subtable}
    \hfill
    \begin{subtable}[t]{.45\textwidth}
        \centering
        \begin{tabularx}{\textwidth}{r R l}
            \toprule
            rank & $n$-gram & distance\\
            \midrule
            \num{0} & logic takes care of itself & \num{0}\\
            \num{1} & keratomileusi for & \num{0.0555444}\\
            \num{2} & unicompartment knee & \num{0.0562137}\\
            \num{3} & the fasttrack & \num{0.0571033}\\
            \num{4} & of mcts & \num{0.0586164}\\
            \num{5} & base lube & \num{0.058743}\\
            \num{6} & pareto rank & \num{0.0592106}\\
            \num{7} & basket peg & \num{0.059318}\\
            \num{8} & to rauschenberg & \num{0.0596529}\\
            \num{9} & carcinoid & \num{0.0600035}\\
            \num{10} & jiceng & \num{0.0602956}\\
            \num{11} & cpe cours & \num{0.0604733}\\
            \num{12} & the tetrodotoxin & \num{0.0604942}\\
            \midrule
            \num{13} & dibartola & \num{0.0606432}\\
            \num{14} & qujiang & \num{0.0610779}\\
            \num{15} & sequenti assembl & \num{0.06159}\\
            \num{16} & engag scholarship & \num{0.0622144}\\
            \num{17} & merrienbo & \num{0.062344}\\
            \num{18} & zhang daol & \num{0.062384}\\
            \num{19} & combust environ & \num{0.0623893}\\
            \bottomrule
        \end{tabularx}
        \caption{\enquote{logic takes care of itself}}\label{tab:ranking_chance_2}
    \end{subtable}
    \caption[Neighbors of longer $n$-grams]{Neighbors: $1+2$-grams, $r = 10$, $[129,256]$}\label{tab:ranking_chance}
\end{table}

The situation changes when we try to query phrases and add $2$-grams to the data set. As \autoref{tab:ranking_chance_1} shows, results may contain more entries that do not provide any insights. The data in \autoref{tab:ranking_chance_2} are even worse. At least for us, no insights are extracted at all.\footnote{People might be able to interpret the \enquote{zhang daol} entry, which is derived from the name \enquote{Zhang Daoling}.} When limiting the search to the $1$-gram database, the results are even worse. Some reasons for this are that our cleaned up data contains a lot of words that do not provide direct insights. For example the words \enquote{the} and \enquote{of}, which are listed in some $2$-gram neighbors, do not provide meaningful data. On the other hand the same words may be important for $3$-grams, e.g., in \enquote{Medal of Honor}. The reason that so many neighbors are found that seem to contain random data is the following: the more data is stored in the database, the more likely it is to find neighbors that have a lower distance. A similar effect can be observed when dealing with correlations of real world data. \cite{correlations1} contains a list of these cases that, with a very high probability, do not belong to the same cause. This is a general issue with methods that try to extract links between words or concepts based on usage count. We were aware of this problem beforehand and it is discussed in \autoref{sec:baseline:sim}. Possible workarounds can be:

\begin{table}[t]
    \begin{subtable}[t]{.45\textwidth}
        \centering
        \begin{tabularx}{\textwidth}{r R l}
            \toprule
            rank & $n$-gram & distance\\
            \midrule
            \num{0} & basic introduct & \num{0.11728}\\
            \num{1} & god beyond & \num{0.117417}\\
            \num{2} & underground chamber & \num{0.117425}\\
            \num{3} & all biblic & \num{0.11759}\\
            \num{4} & stori add & \num{0.117698}\\
            \num{5} & less put & \num{0.118025}\\
            \num{6} & would altern & \num{0.118063}\\
            \num{7} & also proof & \num{0.118091}\\
            \midrule
            \num{8} & for tighter & \num{0.118122}\\
            \num{9} & pay consult & \num{0.118165}\\
            \num{10} & simpli admit & \num{0.118373}\\
            \num{11} & child requir & \num{0.11839}\\
            \num{12} & aloud what & \num{0.118508}\\
            \num{13} & those think & \num{0.118584}\\
            \num{14} & univers scienc & \num{0.118605}\\
            \num{15} & all sixti & \num{0.118674}\\
            \num{16} & isaac in & \num{0.118777}\\
            \num{17} & selfconsci of & \num{0.118791}\\
            \num{18} & also imit & \num{0.118859}\\
            \num{19} & some consult & \num{0.118875}\\
            \bottomrule
        \end{tabularx}
        \caption{\enquote{think therefore i am}}\label{tab:ranking_chance2_1}
    \end{subtable}
    \hfill
    \begin{subtable}[t]{.45\textwidth}
        \centering
        \begin{tabularx}{\textwidth}{r R l}
            \toprule
            rank & $n$-gram & distance\\
            \midrule
            \num{0} & via hematogen & \num{0.099445}\\
            \num{1} & messag share & \num{0.1004}\\
            \num{2} & creatur & \num{0.100961}\\
            \num{3} & fof & \num{0.101037}\\
            \num{4} & hree & \num{0.101637}\\
            \num{5} & consult typic & \num{0.101826}\\
            \midrule
            \num{6} & hypoxia to & \num{0.101862}\\
            \num{7} & irom & \num{0.101975}\\
            \num{8} & of admiss & \num{0.102331}\\
            \num{9} & vate & \num{0.102352}\\
            \num{10} & nli & \num{0.102393}\\
            \num{11} & reflect dietari & \num{0.102469}\\
            \num{12} & hap & \num{0.102618}\\
            \num{13} & economi surg & \num{0.102829}\\
            \num{14} & fold plastic & \num{0.103087}\\
            \num{15} & therapist obtain & \num{0.103215}\\
            \num{16} & possibl point & \num{0.103358}\\
            \num{17} & unlimit amount & \num{0.103394}\\
            \num{18} & cide & \num{0.103399}\\
            \num{19} & paranorm in & \num{0.10344}\\
            \bottomrule
        \end{tabularx}
        \caption{\enquote{logic takes care of itself}}\label{tab:ranking_chance2_2}
    \end{subtable}
    \caption[Neighbors under distance combination]{Neighbors: $1+2$-grams, $r = 10$, $[129,256]$, combined with \enquote{philosophi}}\label{tab:ranking_chance2}
\end{table}

\begin{itemize}
    \item Increasing the pruning threshold: This should eliminate many pointless entries that only do not represent a common trend.
    \item Weighting distance with $n$-gram importance: It should be possible to weight the distance with additional information about possible neighbors, e.g., with their total counts or with linguistic metrics on how common or special words are.
    \item Extend filtering/normalization to take inter-word information into account: Currently prepositions and other non-important words in $2$-grams and at the end of $3+$-grams often do not provide value information.
    \item Use multiple search queries and calculate a (weighted) distance to all of them: An example is shown in \autoref{tab:ranking_chance2} where we add \enquote{philosophi} as a second query. There exist multiple methods on how to do that, in our example we just added the DTW distances together, but Euclidean distances may also work as well as more sophisticated ways like running a DTW against multiple time series at once.
\end{itemize}



\section{Compression Distortion}
\label{sec:evaluation:distortion}

Because our compression method leads to information loss, we need to figure out how that loss affects the data in terms of different use cases. We have build multiple methods to compare the compressed data with the uncompressed one and will describe them in the following subsections.


\subsection{User-facing data}
\label{ssec:evaluation:distortion:user}

\begin{figure}
    \centering
    \begin{tikzpicture}
    \begin{axis}[
        xlabel={compression error threshold},
        legend above,
        xmode=log,
        fullwidth
    ]
        \addplot [mark=+,color=kit-green2] table [x=maxerror, y=sdiff, col sep=comma, ignore chars={"}] {data/wavelet_compression.csv};
        \addplot [mark=o,color=kit-yellow] table [x=maxerror, y=enew, col sep=comma, ignore chars={"}] {data/wavelet_compression.csv};
        \addplot [mark=asterisk,color=kit-red] table [x=maxerror, y=levelx, col sep=comma, ignore chars={"}] {data/wavelet_compression.csv};
        \legend{Levenshtein distance,new entries,\% nodes};
    \end{axis}
\end{tikzpicture}

    \caption[Compression distortion, user facing data]{Compression distortion, user facing data, query is \enquote{drug}}\label{fig:compression}
\end{figure}

First we want to figure out how compression affects the data that is perceived by the user. We use the output as described in \autoref{ssec:baseline:sim:rank} and plug in the optional compression before the smoothing as it was discussed in \autoref{sec:algorithm:wavelet}. We choose a single cutoff point for both, the uncompressed and compressed, results since the choice is very unstable under small compression artifacts. We then calculate the following two metrics:

\begin{enumerate}
    \item We treat the sorted list of the nearest neighbors as strings and calculate the Levenshtein distance (\cite{levenshtein}) of them.
    \item We measure the amount of new entries produced by the compressed version compared to the uncompressed reference as a relative measure.
\end{enumerate}

As shown in \autoref{fig:compression} the set of the nearest neighbors and the ranking of these neighbors is very unstable as soon as we introduce some noise to the data. This makes our algorithm rather bad when the output data is directly presented to the user. This situation might be improve when combining the distance to multiple queries to one final score, which might be required for handling groups of words or entire concepts as a search query.


\subsection{All distances}
\label{ssec:evaluation:distortion:dist}

\begin{figure}
    \centering
    \begin{tikzpicture}
    \begin{axis}[
        xlabel={error threshold},
        ylabel={delta quantile},
        legend above,
        xmode=log,
        ymin=-0.6,
        ymax=0.75,
        extra y ticks={-0.1,0.1},
        extra y tick labels=,
        extra y tick style={
            grid=major
        },
        fullwidth large
    ]
        \addlegendimage{empty legend}
        \addplot [mark=+,color=kit-green] table [x=maxerror, y=min, col sep=comma, ignore chars={"}] {data/wavelet_compression.csv};
        \addplot [mark=o,color=kit-blue] table [x=maxerror, y=q002, col sep=comma, ignore chars={"}] {data/wavelet_compression.csv};
        \addplot [mark=asterisk,color=kit-green2] table [x=maxerror, y=q009, col sep=comma, ignore chars={"}] {data/wavelet_compression.csv};
        \addplot [mark=diamond,color=kit-yellow] table [x=maxerror, y=q025, col sep=comma, ignore chars={"}] {data/wavelet_compression.csv};
        \addplot [mark=oplus,color=kit-brown] table [x=maxerror, y=median, col sep=comma, ignore chars={"}] {data/wavelet_compression.csv};
        \addplot [mark=pentagon,color=kit-violet] table [x=maxerror, y=q075, col sep=comma, ignore chars={"}] {data/wavelet_compression.csv};
        \addplot [mark=x,color=kit-cyan] table [x=maxerror, y=q091, col sep=comma, ignore chars={"}] {data/wavelet_compression.csv};
        \addplot [mark=otimes,color=kit-green!75] table [x=maxerror, y=q098, col sep=comma, ignore chars={"}] {data/wavelet_compression.csv};
        \addplot [mark=square,color=kit-blue!75] table [x=maxerror, y=max, col sep=comma, ignore chars={"}] {data/wavelet_compression.csv};

        \legend{\textbf{quantile:},$0.00$,$0.02$,$0.09$,$0.25$,$0.50$,$0.75$,$0.91$,$0.98$,$1.00$};
    \end{axis}
\end{tikzpicture}

    \caption[Compression distortion, all distances]{Compression distortion, all distances, query is \enquote{drug}}\label{fig:compression2}
\end{figure}

To measure how the compression affects the actual distance, we calculate the distance to a fixed query and calculate the normalized distance of the original result and the ones produced by the compressed data:

\begin{equation}
    \delta_i(a, b) = \frac{b_i - a_i}{a_i}
\end{equation}

Then we measure different quantiles of this distance as plotted in \autoref{fig:compression2}. We also annotated the markers for \SI{-10}{\percent} and \SI{10}{\percent} to show when, in our opinion, the distortion gets unacceptable for different amounts of time series. It turns out that even with some outliers, the overall result is more stable than the user-facing data. Therefore we conclude that for some applications higher compression rates and the corresponding artifacts can be acceptable.



\section{Quality of the Tree-Index}
\label{sec:evaluation:tb}

\begin{figure}
    \centering
    \begin{tikzpicture}
    \newcommand{\mytbplot}[4]{
        \addplot [mark=#3,color=#4,only marks] table [x=ndtw, y=enew, col sep=comma, ignore chars={"}, discard if not 2=maxlevel #1 and maxerror #2] {data/tbindex.csv};
    }

    \newcommand{\mytraces}[1]{
        \foreach \myng in {{democraci},{drug},{german},{happi},{health},{know},{money},{religion},{soft},{war}} {
            \foreach \myr in {0,5,10,15,20} {
                \addplot [mark=none,color=black,draw opacity=0.05,smooth,tension=0.2,forget plot] table [x=ndtw, y=enew, col sep=comma, ignore chars={"}, discard if not 3=maxerror #1 and ngram {\myng} and r \myr] {data/tbindex.csv};
            }
        }
    }

    \begin{groupplot}[
        xlabel={\# DTWs executed},
        ylabel={\# new entries in top 20 neighbors},
        legend above,
        group style={
            group name=my tbplots,
            group size=1 by 3,
            xlabels at=edge bottom,
            vertical sep=1cm
        },
        ymin=0,
        ymax=1,
        xmin=0,
        xmax=803042,
        enlargelimits={rel=0.1},
        fullwidth
    ]
        \nextgroupplot

        \mytraces{1}

        \addlegendimage{empty legend}
        \addlegendentry{\textbf{maximum tracing level:}};

        \mytbplot{0}{1}{+}{kit-green2}
        \addlegendentry{0};

        \mytbplot{1}{1}{x}{kit-brown}
        \addlegendentry{1};

        \mytbplot{2}{1}{square}{kit-yellow}
        \addlegendentry{2};

        \mytbplot{3}{1}{o}{kit-violet}
        \addlegendentry{3};

        \mytbplot{4}{1}{diamond}{kit-cyan}
        \addlegendentry{4};

        \mytbplot{5}{1}{pentagon}{kit-green}
        \addlegendentry{5};

        \mytbplot{6}{1}{oplus}{kit-blue}
        \addlegendentry{6};

        \nextgroupplot

        \mytraces{2}

        \mytbplot{0}{2}{+}{kit-green2}
        \mytbplot{1}{2}{x}{kit-brown}
        \mytbplot{2}{2}{square}{kit-yellow}
        \mytbplot{3}{2}{o}{kit-violet}
        \mytbplot{4}{2}{diamond}{kit-cyan}
        \mytbplot{5}{2}{pentagon}{kit-green}
        \mytbplot{6}{2}{oplus}{kit-blue}

        \nextgroupplot

        \mytraces{5}

        \mytbplot{0}{5}{+}{kit-green2}
        \mytbplot{1}{5}{x}{kit-brown}
        \mytbplot{2}{5}{square}{kit-yellow}
        \mytbplot{3}{5}{o}{kit-violet}
        \mytbplot{4}{5}{diamond}{kit-cyan}
        \mytbplot{5}{5}{pentagon}{kit-green}
        \mytbplot{6}{5}{oplus}{kit-blue}
    \end{groupplot}
\end{tikzpicture}

    \caption[Efficiency of tracer, testing depth filter + compression rate]{Efficiency of traceback-based indexing, testing trace-down depth and maximum compression error}\label{fig:tbindex}
\end{figure}

We now want to evaluate how the index structure described in \autoref{sec:algorithm:asindex} behaves under certain compression levels and tracing limits. The wavelet index is only used to select candidates for the DTW-driven nearest neighbor search. It is not used to provide the actual time series data. This information is retrieved from the uncompressed data set. We doubt that our compression method is good enough to drive both components at the same time, especially when the same maximum error is used.

We set up the following experiment: We use the $n$-grams \enquote{democraci}, \enquote{drug}, \enquote{german} \enquote{happi}, \enquote{health}, \enquote{know}, \enquote{money}, \enquote{religion}, \enquote{soft}, and \enquote{war} and a warping radius of \num{0}, \num{5}, \num{10}, \num{15}, and \num{20}. No weight-based filtering is applied. We then compare the top \num{20} neighbors for all combinations under the set of compression levels and maximum trace-down limits. We compare the number of new entries within that list. If the resulting is shorted than \num{20}, because not enough candidates could be extracted from the index, we do a fill-up with virtual entries that never match any existing one. Without that special handling, only the precision of the index would be measured and low recall values would not affect the score. Also, we do not measure the string distance here since our approach does not influence the pair-wise ranking of the correctly drawn neighbors. This is due to the fact that all candidates returned by the index are compared via DTW without any priority handling.

The results are shown in \autoref{fig:tbindex}. The top graphics shows the index under a maximum compression error of \num{1}, the middle one with an error of \num{2} and the bottom one with a value of \num{5}. The different plotting styles represent different maximum tracing levels. Sets of markers that share the same x-coordinate usually have their origin in the same warping radius since our tracing algorithm does not take this variable into account. The different groups of markers then correspond to different queries. Measurements that belong to the same query and warping radius, hence the same information request, are connected with a thin line to simplify tracking of the parameter response. It can be observed that different queries are affected differently by the index while the influence of different warping radius values is not that big. Another aspect is the choice of the right compression level. A maximum error of \num{1} makes it hard to draw enough candidates from the index while an error of \num{5} results in too many results and therefore does not provide enough value. We therefore assume that a maximum error of around \num{2} results in good results and we will use this index for further measurements.

\begin{figure}
    \centering
    \begin{tikzpicture}
    \newcommand{\mytbplot}[3]{
        \addplot [mark=#2,color=#3,only marks] table [x=ndtw, y=enew, col sep=comma, ignore chars={"}, discard if not={minweight}{#1}] {data/tbindex2.csv};
    }

    \begin{axis}[
        xlabel={\# DTWs executed},
        ylabel={\# new entries in top 20 neighbors},
        legend above,
        fullwidth
    ]
        \foreach \myng in {{democraci},{drug},{german},{happi},{health},{know},{money},{religion},{soft},{war}} {
            \foreach \myr in {0,5,10,15,20} {
                \addplot [mark=none,color=black,draw opacity=0.1,smooth,tension=0.5,forget plot] table [x=ndtw, y=enew, col sep=comma, ignore chars={"}, discard if not 2=ngram {\myng} and r \myr] {data/tbindex2.csv};
            }
        }

        \mytbplot{0.0000000000001}{o}{kit-red}
        \addlegendentry{\num{1e-13}};

        \mytbplot{0.000000000001}{+}{kit-green2}
        \addlegendentry{\num{1e-12}};

        \mytbplot{0.00000000001}{x}{kit-brown}
        \addlegendentry{\num{1e-11}};

        \mytbplot{0.0000000001}{square}{kit-violet}
        \addlegendentry{\num{1e-10}};

        \mytbplot{0.000000001}{diamond}{kit-cyan}
        \addlegendentry{\num{1e-9}};

        \mytbplot{0.00000001}{pentagon}{kit-green}
        \addlegendentry{\num{1e-8}};

        \mytbplot{0.0000001}{oplus}{kit-blue}
        \addlegendentry{\num{1e-7}};

        \mytbplot{0.000001}{asterisk}{kit-red}
        \addlegendentry{\num{1e-6}};
    \end{axis}
\end{tikzpicture}

    \caption[Efficiency of tracer, testing minimum weight filter]{Efficiency of traceback-based indexing, testing minimum weight filter}\label{fig:tbindex2}
\end{figure}

Sometimes using the compression rate as hint how exact the nearest neighbor search should be might not be sufficient, for example if the index is already created. Luckily our generic tracer approach allows us to incorporate weight-based filtering. We used that to retrieve data from the index with a maximum error of \num{2} faster in trade-off against the quality of the results. How the minimum weight threshold affects the quality is shown in \autoref{fig:tbindex2}. The setup is identical to \autoref{fig:tbindex}. It turns out that the impact of the filter varies heavily from query to query so it may be hard for users to guess the right value for their expected results. Also, the filter response is not monotonic in neither result quality nor number of executed DTWs. Further research is required to make the filter results more predictable.



\section{Performance}
\label{sec:evaluation:perf}

\begin{table}[t]
    \centering
    \begin{tabularx}{\textwidth}{r L}
        \toprule
        paremeter & value\\
        \midrule
        CPU & \nohyphens{Intel\circledR{} Core\textsuperscript{TM} i5-5200U at \SI{2.2}{\giga\hertz}}\\
        RAM & \nohyphens{\num{2} times \SI{4}{\giga\byte} DDR3 at \SI{1600}{\mega\hertz}}\\
        SSD & \nohyphens{Samsung\circledR{} MZNLN256, \SI{256}{\giga\byte}}\\
        \midrule
        CPU Microcode & \nohyphens{\code{0x22}}\\
        OS & \nohyphens{Arch Linux}\\
        Kernel & \nohyphens{Linux \code{4.6.3}}\\
        Compiler & \nohyphens{Clang \code{3.8.0}}\\
        Linker & \nohyphens{GNU gold \code{1.11}}\\
        C-lib & \nohyphens{glibc \code{2.23}}\\
        \Cpp{}-lib & \nohyphens{libstdc++ \code{6.1.1}}\\
        Filesystem & \nohyphens{Btrfs with \code{rw, relatime, ssd, space\_cache, discard, commit=60, autodefrag, compress=lzo}}\\
        \midrule
        Relevant compiler parameters & \nohyphens{\code{\mbox{-fPIC} \mbox{-fdata-sections} \mbox{-ffunction-sections} \mbox{-g} \mbox{-fno-omit-frame-pointer} \mbox{-mavx2} \mbox{-mfma} \mbox{-fsized-deallocation} \mbox{-std=c++14} \mbox{-ffast-math} \mbox{-pthread} \mbox{-O3} \mbox{-fuse-ld=gold} \mbox{-Wl,-\phantom{}-disable-new-dtags} \mbox{-static} \mbox{-Wl,-\phantom{}-gc-sections} \mbox{-Wl,-\phantom{}-no-export-dynamic}}}\\
        \bottomrule
    \end{tabularx}
    \caption{Performance testing environment}\label{tab:env}
\end{table}

Finally, we want to figure out how fast our algorithm is compared to the baseline. To do so, we carry out the following measurements: On a system described in \autoref{tab:env}, we carry out \num{20} executions and measure the average real time of them. We decided against measuring only user space time because that would miss out IO operations of our memory mapped data storage system.

\begin{figure}
    \centering
    \begin{tikzpicture}
    \begin{axis}[
        xlabel={$r$},
        ylabel={t},
        legend above,
        fullwidth
    ]
        \addplot [mark=*,color=kit-green2] table [x=r, y=t, col sep=comma, ignore chars={"}, discard if not={ngram}{drug}] {data/perf_qsimple_r.csv};
        \addplot [mark=*,color=kit-cyan] table [x=r, y=t, col sep=comma, ignore chars={"}, discard if not={ngram}{know}] {data/perf_qsimple_r.csv};
        \addplot [mark=*,color=kit-red] table [x=r, y=t, col sep=comma, ignore chars={"}, discard if not={ngram}{war}] {data/perf_qsimple_r.csv};
        \addplot [mark=x,color=kit-green2] table [x=r, y=t, col sep=comma, ignore chars={"}, discard if not={ngram}{drug}] {data/perf_qindexed_r.csv};
        \addplot [mark=x,color=kit-cyan] table [x=r, y=t, col sep=comma, ignore chars={"}, discard if not={ngram}{know}] {data/perf_qindexed_r.csv};
        \addplot [mark=x,color=kit-red] table [x=r, y=t, col sep=comma, ignore chars={"}, discard if not={ngram}{war}] {data/perf_qindexed_r.csv};
        \legend{drug (no index),know (no index),war (no index),drug (index),know (index),war (index)};
    \end{axis}
\end{tikzpicture}

    \caption{Time of baseline algorithms over $r$}\label{fig:perf_baseline_r}
\end{figure}

The first tests apply to the baseline. As shown in \autoref{fig:perf_baseline_r}, the runtime of na{\"\i}ve approach grows with increasing values of $r$. The index-based method keeps this growing at a minimum but starts at a higher initial cost. The break-even point is at around $r = 5$, slightly depending on the query. Also, keep in mind that every warping radius needs its own prepared index. We want to point out that every run has its initial boot-up time for loading the data, even when most data is stored in memory mapped files. But that costs should be the same, no matter what type of query is run and what acceleration method is used.

\begin{figure}
    \centering
    \begin{tikzpicture}
    \begin{axis}[
        xlabel={$r$},
        ylabel={t in seconds},
        legend above cols=3,
        fullwidth
    ]
        \addplot [mark=+,color=kit-green2] table [x=r, y=t, col sep=comma, ignore chars={"}, discard if not={ngram}{drug}] {data/perf_qwavelet_me2_md4_mw0_r.csv};
        \addplot [mark=+,color=kit-cyan] table [x=r, y=t, col sep=comma, ignore chars={"}, discard if not={ngram}{know}] {data/perf_qwavelet_me2_md4_mw0_r.csv};
        \addplot [mark=+,color=kit-red] table [x=r, y=t, col sep=comma, ignore chars={"}, discard if not={ngram}{war}] {data/perf_qwavelet_me2_md4_mw0_r.csv};

        \addplot [mark=x,color=kit-green2] table [x=r, y=t, col sep=comma, ignore chars={"}, discard if not={ngram}{drug}] {data/perf_qwavelet_me2_md5_mw0_r.csv};
        \addplot [mark=x,color=kit-cyan] table [x=r, y=t, col sep=comma, ignore chars={"}, discard if not={ngram}{know}] {data/perf_qwavelet_me2_md5_mw0_r.csv};
        \addplot [mark=x,color=kit-red] table [x=r, y=t, col sep=comma, ignore chars={"}, discard if not={ngram}{war}] {data/perf_qwavelet_me2_md5_mw0_r.csv};

        \addplot [mark=o,color=kit-green2] table [x=r, y=t, col sep=comma, ignore chars={"}, discard if not={ngram}{drug}] {data/perf_qwavelet_me2_md5_mw0.0000001_r.csv};
        \addplot [mark=o,color=kit-cyan] table [x=r, y=t, col sep=comma, ignore chars={"}, discard if not={ngram}{know}] {data/perf_qwavelet_me2_md5_mw0.0000001_r.csv};
        \addplot [mark=o,color=kit-red] table [x=r, y=t, col sep=comma, ignore chars={"}, discard if not={ngram}{war}] {data/perf_qwavelet_me2_md5_mw0.0000001_r.csv};

        \addplot [mark=square,color=kit-green2] table [x=r, y=t, col sep=comma, ignore chars={"}, discard if not={ngram}{drug}] {data/perf_qwavelet_me2_md5_mw0.0000000000001_r.csv};
        \addplot [mark=square,color=kit-cyan] table [x=r, y=t, col sep=comma, ignore chars={"}, discard if not={ngram}{know}] {data/perf_qwavelet_me2_md5_mw0.0000000000001_r.csv};
        \addplot [mark=square,color=kit-red] table [x=r, y=t, col sep=comma, ignore chars={"}, discard if not={ngram}{war}] {data/perf_qwavelet_me2_md5_mw0.0000000000001_r.csv};

        \legend{
            {drug (depth $=$ \num{4}, weight $=$ \num{0})},
            {know (depth $=$ \num{4}, weight $=$ \num{0})},
            {war (depth $=$ \num{4}, weight $=$ \num{0})},
            {drug (depth $=$ \num{5}, weight $=$ \num{0})},
            {know (depth $=$ \num{5}, weight $=$ \num{0})},
            {war (depth $=$ \num{5}, weight $=$ \num{0})},
            {drug (depth $=$ \num{5}, weight $=$ \num{1e-13})},
            {know (depth $=$ \num{5}, weight $=$ \num{1e-13})},
            {war (depth $=$ \num{5}, weight $=$ \num{1e-13})},
            {drug (depth $=$ \num{5}, weight $=$ \num{1e-7})},
            {know (depth $=$ \num{5}, weight $=$ \num{1e-7})},
            {war (depth $=$ \num{5}, weight $=$ \num{1e-7})}
        };
    \end{axis}
\end{tikzpicture}

    \caption{Time of wavelet-index algorithms over $r$}\label{fig:perf_ours_r}
\end{figure}

The next tests belong to our wavelet-based index. Keep in mind that this index does not provide accurate results as shown in the previous section and that the wavelet structure is only used for candidate selection, not for extracting the actual time series data. We selected a subset of possible configurations, namely a maximum tracing level of \num{4} with no minimum weight filter and a maximum level of \num{5} with a filter of \num{0} (no filter), \num{1e-7}, and \num{1e-13}. The results are presented in \autoref{fig:perf_ours_r}. We can report that we can accelerate the DTW queries if missing neighbors are an acceptable solution. This can be the case if someone wants only a peek at the data or if algorithms can handle missing links, for example during graph processing.

These results are, to be honest, not convincing when using the algorithm for a general-purpose user-facing application but they might be useful for special algorithms and use-cases.
